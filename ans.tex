\documentclass{article}

\usepackage{amsmath}
\usepackage{amssymb}
\usepackage{enumerate}
\usepackage[margin=2cm]{geometry}
\usepackage{graphicx}

\newcommand{\R}{\mathbb{R}}
\newcommand{\tuple}[1]{\langle #1 \rangle}
\newcommand{\centeredimage}[1]{\begin{center}\includegraphics{#1}\end{center}}

\title{Theoretical Physics (G. Joos) - Exercises}
\author{Vedant Kumar}

\begin{document}
\maketitle

\section{Vector Analysis}

\begin{enumerate}[1)]
    \item Express that three vectors with the same orientation form a closed
        triangle.
        
        $A + B + C = 0$. 
    \item
        \begin{enumerate}[a)]
            \item Check that two vectors are parallel.
                
                $\frac{A}{|A|} = \frac{B}{|B|}$. Since the origin is
                invariant, only scale and direction matter.
            \item Check that three vectors are coplanar (assume $\R^3$, and
                that the vectors are not parallel).
                
                Some linear combination of $A$ and $B$ must give $C$, since
                the subspace spanned by the plane has dimension 2:
                $aA + bB = C$. 
        \end{enumerate}

    \item Give the geometric significance of
        $(A+B)^2 = A^2 + B^2 + 2(A \cdot B)$.

        \centeredimage{p1-3}

    \item What is $(A+B) \cdot (A-B)$ when $A^2=B^2$?

        $A^2 - AB + AB - B^2 = 0$, so the two terms are orthogonal.

    \item Calculate the angle b/w $S_n = \tuple{\cos x_n, \cos y_n, \cos
        z_n}$, $n \in \{1, 2\}$.

        \begin{align*}
            S_1 \cdot S_2 &=
                 \cos x_1\cos x_2 + \cos y_1\cos y_2 + \cos z_1\cos z_2 \\
                          &= |S_1||S_2|\cos \alpha \\
                          &= \sqrt{(\cos^2 x_1 + \cos^2 y_1 + \cos^2 z_1)}
                             \sqrt{(\cos^2 x_2 + \cos^2 y_2 + \cos^2 z_2)}
                             \cos \alpha \\
            \alpha &= \cos^{-1}\left( \frac{
                \cos x_1\cos x_2 + \cos y_1\cos y_2 + \cos z_1\cos z_2
            }{
                \sqrt{(\cos^2 x_1 + \cos^2 y_1 + \cos^2 z_1)}
                \sqrt{(\cos^2 x_2 + \cos^2 y_2 + \cos^2 z_2)}
            } \right)
        \end{align*}

    \item Verify: $A \times (B+C) = A \times B + A \times C$. 

        \begin{align*}
            A \times B &= \tuple{(A_yB_z - A_zB_y), (A_zB_x - A_xB_z),
                                 (A_xB_y - A_yB_x)} \\
            A \times C &= \tuple{(A_yC_z - A_zC_y), (A_zC_x - A_xC_z),
                                 (A_xC_y - A_yC_x)} \\
            A \times B + A \times C &= \tuple{A_y(B_z+C_z) - A_z(B_y+C_y),
                                              A_z(B_x+C_x) - A_x(B_z+C_z),
                                              A_x(B_y+C_y) - A_y(B_y+C_y)} \\
                       &= A \times (B+C)
        \end{align*}

    \item What is the value of $(A \times B)^2 + (A \cdot B)^2$?

        \begin{align*}
            (A \times B)^2 + (A \cdot B)^2 &=
            |A \times B|^2 + |A \cdot B|^2 \\
            &= (|A||B|\sin \theta)^2 + (|A||B|\cos \theta)^2 \\
            &= |A|^2|B|^2(\sin^2 \theta + \cos^2 \theta) \\
            &= |A|^2|B|^2
        \end{align*}

    \item
        \begin{enumerate}
            \item Give the plane equation given some point $p_0$ on the
                plane and the normal:

                $n \cdot (p - p_0) = 0$.
            \item Give the distance of a point $r_0$ from a plane:

                The shortest path from $r_0$ to the plane is given by the
                projection of $p_0 - r_0$ onto the axis defined by the
                normal, i.e $n \cdot (p_0 - r_0)$.

                Here's another way to look at this. We can start at $r_0$
                and move along an axis defined by the plane normal to take the
                shortest path towards the plane. Assuming a unit normal, we hit
                the plane when:
                \begin{align*}
                    0 &= ((r_0 + \alpha n) - p_0) \cdot n \\
                    0 &= (r_0 + \alpha n) \cdot n - p_0 \cdot n \\
                    0 &= r_0 \cdot n + \alpha - p_0 \cdot n \\
                    \alpha &= n \cdot (p_0 - r_0)
                \end{align*}
            \item Define a plane given three points (not all collinear):

                All points $p$ s.t
                $[(r_1 - r_0) \times (r_2 - r_0)](p - r_0) = 0$.
        \end{enumerate}

    \item Simplify: $(A \times B) \times (C \times D)$.

        Let $E = A \times B$. Then:
        \begin{align*}
            E \times (C \times D) = C(E \cdot D) - D(E \cdot C)
        \end{align*}

        Equivalently we can write $F = C \times D$ and get:
        \begin{align*}
            (A \times B) \times F &= -F \times (A \times B) \\
                                  &= A(-F \cdot B) - B(-F \cdot A) \\
                                  &= B(A \cdot F) - A(B \cdot F)
        \end{align*}

    \item When is $A \times \frac{dA}{du} = 0$?

        When $A$ is constant and/or parallel to $\frac{dA}{du}$.

    \item Imagine a surface $A$ (e.g, the shell of a sphere) and a boundary
        curve $R$ of $A$ restricted to a plane (e.g, its equator). Show that
        $\int_R (b \times r) dR = 0$.

        In this case, imagine that the sphere is centered at the origin. The
        binormal at a point $r$ on the equator points `up' ($\hat{k}$) and
        $r$ itself points `outwards'. The cross product is just the tangent
        vector: integrated around a closed curve, this gives 0.
\end{enumerate}

\end{document}
