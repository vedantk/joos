\documentclass{article}

\usepackage{amsmath}
\usepackage{amssymb}
\usepackage{enumerate}
\usepackage[margin=2cm]{geometry}
\usepackage{graphicx}

\newcommand{\R}{\mathbb{R}}
\newcommand{\tuple}[1]{\langle #1 \rangle}
\newcommand{\centeredimage}[1]{\begin{center}\includegraphics{#1}\end{center}}

\title{Theoretical Physics (G. Joos) - Exercises}
\author{Vedant Kumar}

\begin{document}
\maketitle

\section{Vector Analysis}

\begin{enumerate}[1)]
    \item Express that three vectors with the same orientation form a closed
        triangle.
        
        $A + B - C = 0$. This is a little strange, because you have to
        imagine each vector as a ray with an origin.
    \item
        \begin{enumerate}[a)]
            \item Check that two vectors are parallel.
                
                $\frac{A}{|A|} = \frac{B}{|B|}$. Since the origin is
                invariant, only scale and direction matter.
            \item Check that three vectors are coplanar (assume $\R^3$, and
                that the vectors are not parallel).
                
                Some linear combination of $A$ and $B$ must give $C$, since
                the subspace spanned by the plane has dimension 2:
                $aA + bB = C$. 
        \end{enumerate}
    \item Give the geometric significance of $(A+B)^2 = A^2 + B^2 + 2AB$.

        \centeredimage{p1-3}
    \item What does $(A+B)(A-B)$ mean when $A^2=B^2$?

        Nothing. $A^2 - AB + AB - B^2 = 0$.
    \item Calculate the angle b/w $S_n = \tuple{\cos x_n, \cos y_n, \cos
        z_n}$, $n \in \{1, 2\}$.

        \begin{align*}
            S_1S_2 &= \cos x_1\cos x_2 + \cos y_1\cos y_2 + \cos z_1\cos z_2
                      \\
                   &= |S_1||S_2|\cos \alpha \\
                   &= \sqrt{(\cos^2 x_1 + \cos^2 y_1 + \cos^2 z_1)}
                      \sqrt{(\cos^2 x_2 + \cos^2 y_2 + \cos^2 z_2)}
                      \cos \alpha \\
            \alpha &= \cos^{-1}\left( \frac{
                \cos x_1\cos x_2 + \cos y_1\cos y_2 + \cos z_1\cos z_2
            }{
                \sqrt{(\cos^2 x_1 + \cos^2 y_1 + \cos^2 z_1)}
                \sqrt{(\cos^2 x_2 + \cos^2 y_2 + \cos^2 z_2)}
            } \right)
        \end{align*}
    \item Verify: $[A(B+C)] = [AB] + [AC]$. 

        \begin{align*}
            [AB] &= \tuple{(A_yB_z - A_zB_y), (A_zB_x - A_xB_z), (A_xB_y -
            A_yB_x)} \\
            [AC] &= \tuple{(A_yC_z - A_zC_y), (A_zC_x - A_xC_z), (A_xC_y -
            A_yC_x)} \\
            [AB] + [AC] &= \tuple{A_y(B_z+C_z) - A_z(B_y+C_y),
                                  A_z(B_x+C_x) - A_x(B_z+C_z),
                                  A_x(B_y+C_y) - A_y(B_y+C_y)} \\
                        &= [A(B+C)]
        \end{align*}
    \item What is the value of $[AB]^2 + (AB)^2$?

        It's $|A|^2|B|^2\sin^2\theta + |A|^2|B|^2\cos^2\theta$, i.e just
        $A^2B^2$.

    \item
        \begin{enumerate}
            \item Give the plane equation given some point $p_0$ on the
                plane and the normal:

                $n(p - p_0) = 0$.
            \item Give the distance of a point $r_0$ from a plane:

                The shortest path from $r_0$ to the plane is given by the
                projection of $p_0 - r_0$ onto the axis defined by the
                normal, i.e $n(p_0 - r_0)$.

                Here's another way to look at this. We can start at $r_0$
                and move along an axis defined by the plane normal to take the
                shortest path towards the plane. Assuming a unit normal, we hit
                the plane when:
                \begin{align*}
                    0 &= ((r_0 + \alpha n) - p_0)n \\
                    0 &= (r_0 + \alpha n)n - p_0n \\
                    0 &= r_0n + \alpha - p_0n \\
                    \alpha &= n(p_0 - r_0)
                \end{align*}
            \item Define a plane given three points (not all collinear):

                $[(r_1 - r_0)(r_2 - r_0)](p - r_0) = 0$.
        \end{enumerate}
    \item Simplify: $[[AB][CD]]$.

        Let $E = [AB]$. Then:
        \begin{align*}
            [E[CD]] &= C(ED) - D(EC) \\
                    &= C([AB]D) - D([AB]C)
        \end{align*}

        Equivalently we can write $F = [CD]$ and get:
        \begin{align*}
            [[AB]F] &= F(AB) - B(AF) \\
                    &= [CD](AB) - B(A[CD])
        \end{align*}
\end{enumerate}

\end{document}
